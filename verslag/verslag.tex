\documentclass[11pt,a4paper]{article}
\usepackage{graphicx}
\usepackage{sidecap}
\usepackage{mathtools}
%Om grotere integralen te krijgen
\usepackage{relsize}

%Andere breedte en lengthe van een document
\setlength{\textwidth}{6in} 
\addtolength{\hoffset}{-0.5in}
\setlength{\topmargin}{-0.2in}
\setlength{\textheight}{9in}

%Packages voor de figuren:
\usepackage{wrapfig}
\usepackage{caption}
\usepackage{subcaption}
%Kan er voor zorgen dat een figuur op de exacte plaats staat:
\usepackage{float}

%Package voor algoritmen:
\usepackage{algpseudocode}
\usepackage{algorithm}
\begin{document}
\begin{titlepage}

\title{\Huge Toepassingen van meetkunde in de informatica}

\author{Robin Goots\\
		Ward Schodts\\
		}

\date{2013 - 2014}
\maketitle
\thispagestyle{empty}


\begin{center}
\Large Professor Dirk Roose
\vfill
\end{center}
\end{titlepage}

\section{Hoogniveau beschrijving van de algoritmen}

\begin{algorithm}
\caption{eenvoudig algoritme met rekencomplexiteit $O(N^2)$}
\begin{algorithmic}
\State Lijst L met alle cirkels

\While {Niet leeg L}
	\State Cirkel = L.NeemEnVerwijder
	\For {c in L}
	\State snijptn = Cirkel.BerekenSnijpunten(c)
	\State output.VoegToe(snijptn)
	\EndFor
\EndWhile

\Return output
\end{algorithmic}
\end{algorithm}
Voor het volgende algoritme stellen we elke cirkel voor als een lijnstuk met als links eindpunt het meest linkse punt van de cirkel en als rechst eindpunt het meest rechtse eindpunt van de cirkel.
\begin{algorithm}
\caption{doorlooplijnalgoritme met rekencomplexiteit $O(N^2)$}
\begin{algorithmic}
\State Lijst L: met alle punten, gesorteerd op het laagste x-co\"ordinaat daarna op het laagste y-co\"ordinaat en daarna komen linkse punten van een segment voor rechtse punten.
\State Lijst C: met alle cirkels op een bepaald moment van een eventpoint, ongesorteerd.
\For {elk punt p in L}
	\If {p is het linkse eindpunt van een cirkel c}
	
		\State snijptn = Cirkel.BerekenSnijpunten(c)
		\State output.VoegToe(snijptn)
	
	\State VoegToe(C,c)
	\EndIf
	\If {p is het rechtse eindpunt van een cirkel c}
	\State Verwijder(C,c)
	\EndIf
\EndFor

\Return output
\end{algorithmic}
\end{algorithm}
\begin{algorithm}
\caption{complex doorlooplijnalgoritme met rekencomplexiteit $O((N+S)Log(N))$}
\begin{algorithmic}
\State Lijst L: met alle punten, gesorteerd op het laagste x-co\"ordinaat daarna op het laagste y-co\"ordinaat en daarna komen linkse punten van een segment voor rechtse punten.
\State Lijst S: met alle segmenten op een bepaald moment van een eventpoint, ongesorteerd.
\For {elk punt p in de gesorteerde lijst}
	\If {p is het linkse eindpunt van een cirkel c}
	
		\State snijptn = Cirkel.BerekenSnijpunten(c)
		\State output.VoegToe(snijptn)
	
	\State VoegToe(C,c)
	\EndIf
	\If {p is het rechtse eindpunt van een cirkel c}
	\State Verwijder(C,c)
	\EndIf
\EndFor

\Return output
\end{algorithmic}
\end{algorithm}
\end{document}