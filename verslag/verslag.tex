\documentclass[11pt,a4paper]{article}
\usepackage{graphicx}
\usepackage{sidecap}
\usepackage{mathtools}
%Om grotere integralen te krijgen
\usepackage{relsize}
\usepackage[dutch]{babel}

%Andere breedte en lengthe van een document
\setlength{\textwidth}{6in}
\addtolength{\hoffset}{-0.5in}
\setlength{\topmargin}{-0.2in}
\setlength{\textheight}{9in}

%Packages voor de figuren:
\usepackage{wrapfig}
\usepackage{caption}
\usepackage{subcaption}
%Kan er voor zorgen dat een figuur op de exacte plaats staat:
\usepackage{float}

%Package voor algoritmen:
\usepackage{algpseudocode}
\usepackage{algorithm}
\begin{document}
\begin{titlepage}

\title{\Huge Toepassingen van meetkunde in de informatica}

\author{Robin Goots\\
Ward Schodts\\
}

\date{2013 - 2014}
\maketitle
\thispagestyle{empty}


\begin{center}
\Large Professor Dirk Roose
\vfill
\end{center}
\end{titlepage}
\section{De onderzoeksvraag}
Hoe kunnen we de snijpunten van N cirkels vinden?
\begin{enumerate}
\item Wat zijn de verschillende algoritmen?
\item Wat is de rekencomplexiteit van deze algoritmen?
\item In welke omstandigheden presteert het ene algoritme beter als het andere?
\end{enumerate}
\section{De algoritmen}
\subsection{Algoritme 1}
\subsubsection{Algemeen}
Het eerste algortime is een brute-force algoritme. In essentie neemt het \'e\'en van de N cirkels en gaat het dan deze cirkel vergelijken met elke andere cirkel. Om zo te controleren of de cirkels overlappen en dusdanig \'e\'en of twee punten gemeenschappelijk hebben.
\subsubsection{Hoogniveau beschrijving}
\begin{algorithm}
\caption{eenvoudig algoritme met rekencomplexiteit $O(N^2)$}
\begin{algorithmic}
\State Lijst L met alle cirkels
\State Lijst ouput (met alle berekende snijpunten tussen de cirkels) = $\emptyset$
\While {Niet leeg L}
\State Cirkel = L.NeemEnVerwijder()
\For {c in L}
\State snijptn = Cirkel.BerekenSnijpunten(c)
\State output.VoegToe(snijptn)
\EndFor
\EndWhile

\Return output
\end{algorithmic}
\end{algorithm}
De functie \verb|NeemEnVerwijder()| geeft het eerste element van de lijst terug en verwijdert het daarna.\\
De functie \verb|BerekenSnijpunten(Cirkel c)| neemt als argument een cirkel aan en berekent de snijpunten met deze cirkel op de een andere cirkel waar deze methode werd uitgevoerd.

\subsubsection{Complexiteit}
De complexiteit van het algoritme kan theoretisch gevonden worden op de volgende manier: er zijn $N$ cirkels, de eerste wordt gecontroleerd (op snijpunten) met $N-1$ anderen, de volgende met $N-2$ anderen, enzovoort. Dit geeft dus het volgende aantal controles:
$$(N-1) + (N-2) + (N-3) + (N - 4) \cdots $$
Dit is de som van de eerste $N-1$ getallen:
$$ = \frac{N\cdot (N-1)}{2} $$
Dit komt dus overeen met een rekencomplexiteit $O(N^2)$
\subsubsection{Randgevallen}
Voor dit algoritme leggen we als voorwaarde op dat er geen twee cirkels exact op elkaar mogen liggen. Dan zijn er oneindig veel snijpunten.

\subsection{Algoritme 2}
\subsubsection{Algemeen}
Voor het volgende algoritme stellen we elke cirkel voor als een lijnstuk met als links eindpunt het meest linkse punt van de cirkel en als rechts eindpunt het meest rechtse eindpunt van de cirkel. Deze twee punten worden bij in het object cirkel opgelsagen. Al deze $2\cdot N$ punten worden gesorteerd op laagste x-coördinaat, daarna op laagste y-coördinaat, en uiteindelijk op het feit of dat het een links uiterste rechts uiterste is van de cirkel.
\\
\\
Dit algoritme gebruikt een doorlooplijn. De gesorteerde lijst wordt afgegaan, telkens wanneer we een links punt van een cirkel tegenkomen voegen we dit toe aan een tweede lijst. Indien we een rechts punt tegenkomen wordt dit punt uit deze tweede lijst verwijderd. Vanaf de moment dat er een cirkel wordt toegevoegd aan de tweede lijst dan wordt deze cirkel met alle andere cirkels in de lijst vergeleken om te controleren of er geen snijpunten zijn.

\begin{figure}[H]
\centering
\includegraphics[width=0.65\textwidth]{algo2.png}
\end{figure}
In bovenstaande figuur kunnen we het algoritme in werking zien. De rode lijn is de doorlooplijn. Bij punt 1 wordt voor de eerste maal iets toegevoegd, op punt twee wordt een cirkel uit lijst C gehaald. Bij het toevoegen van lijnstuk E in punt 3 wordt er gezocht naar snijpunten met B en D.
\subsubsection{Hoogniveau beschrijving}
\begin{algorithm}[H]
\caption{doorlooplijnalgoritme met rekencomplexiteit $O(N^2)$}
\begin{algorithmic}
\State Lijst L: een gesorteerde lijst met $2N$ punten.
\State Lijst C: met alle cirkels op een bepaald moment van een eventpoint, ongesorteerd.


\For {elk punt p in L}
\If {p is het linkse eindpunt van een lijnstuk c}

\State snijptn = Cirkel.BerekenSnijpunten(c)
\State output.VoegToe(snijptn)
\State VoegToe(C,c)

\Else
\State Verwijder(C,c)
\EndIf
\EndFor

\Return output
\end{algorithmic}
\end{algorithm}
Het punt waar de doorlooplijn zich bevindt wordt een evenpoint genoemd.\\
De methode \verb|BerekenSnijpunten(Cirkel c)| is net hetzelfde als in algortime 1.
\subsubsection{Complexiteit}
De complexiteit wordt wederom bepaald door het aantal keer dat we twee cirkels controleren of ze al dan niet snijden. Aangezien deze controle enkel gebeurt bij linkse punten van een lijnstuk dat een cirkel voorstelt, zijn er N controles (uit de lijst van $2\cdot N$ elementen). Bij elke toevoeging van een cirkel op het moment van een eventpoint wordt deze cirkel vergeleken met alle andere cirkels die op dat moment worden bijgehouden door de sweepline in lijst C. In het slechste geval krijgen we dus een scenario waarbij alle rechterpunten van de lijnstukken samenvallen. Hierdoor moeten we opnieuw net zoals in algoritme 1 $$\frac{N*(N-1)}{2}$$ vergelijkingen maken aangezien er nooit een lijnstuk kan worden verwijderd uit lijst C. Het probleem wordt goed geïllustreerd door onderstaande afbeelding.
\begin{figure}[H]
\centering
\includegraphics[width=0.5\textwidth]{algo2_probleem.png}
\end{figure}
\subsubsection{Randgevallen}
Voor deze implementatie ondersteunen we bijna elke combinatie van de volgende randgevallen:
\begin{enumerate}
\item Dezelfde x-co\"ordinaat
\item Dezelfde y-co\"ordinaat
\item Dezelfde parter (het andere punt van het lijnstuk dat bij de cirkel hoort)
\end{enumerate}
De enige combinatie die we niet ondersteunen is een punt dat dezelfde co\"ordinaten heeft en dezelfde partner. Wat er op neer komt dat het over dezelfde cirkel gaat. Dit zou net zoals in algoritme 1 weer oneindig veel oplossingen geven.
\subsection{Algoritme 3}
\subsubsection{Algemeen}

Voor dit algoritme hebben we een benadering gezocht op basis van een algoritme uit Algorithms Fourth Editions van Princeton dat voor N rechthoeken het aantal snijpunten berekent. Het idee is dat we de cirkels als vierkanten gaan voorstellen. Het algoritme gelijkt zeer fel op dat van algoritme 2. Opnieuw worden alle punten gesorteerd, exact hetzelfde zoals bij algoritme 2. Hierna gebruiken we opnieuw een doorlooplijn die door al deze punten gaat. In plaats van telkens wanneer we een linkerpunt tegenkomen dit te controleren tegen alle andere punten gaan we nu niet meer checken tegen al die andere cirkels die op dat moment ook op de sweepline liggen.\\
\\
We gebruiken nu een binaire zoekboom die gebaseerd is op intervallen, een intervaltree genaamd. De intervallen worden gedefiniëerd door de y-co\"ordinaat van het linkse punt van een lijnstuk vermeerderd en verminderd met de straal. Een voorbeeld is te zien op volgende afbeelding. Op punt 1 op de x-as wordt lijnstuk A toegevoegd aan de intervaltree, op punt twee wordt dit terug verwijderd.
\\
\\
De boom wordt gesorteerd op laagste y-co\"ordinaat.
\begin{figure}[H]
\centering
\includegraphics[width=0.65\textwidth]{algo3.png}
\end{figure}

\subsubsection{Hoogniveau beschrijving}
\begin{algorithm}[H]
\caption{complex doorlooplijnalgoritme met rekencomplexiteit $O((N+S)Log(N))$}
\begin{algorithmic}
\State Lijst L: een gesorteerde lijst met $2N$ punten.
\State Intervaltree T: met alle intervallen op een bepaald moment van een eventpoint, ongesorteerd.
\For {elk punt p in de gesorteerde lijst}
\If {p is het linkse eindpunt van een cirkel c}
\State List mogelijkeSnijpunten = T.checkOverlappingen(p.interval)
	\For {t in mogelijkeSnijpunten}
	\State snijptn = t.cirkel.BerekenSnijpunten(c)
	\State output.VoegToe(snijptn)
	\State VoegToe(T,t)
	\EndFor
\Else
\State Verwijder(T,t)
\EndIf
\EndFor
\Return output
\end{algorithmic}
\end{algorithm}
De methode \verb|checkOverlappingen(Interval t)| moet als argument een interval meekrijgen. Ze kan uitgevoerd worden op een intervaltree, ze geeft dan een lijst van alle nodes in de intervaltree terug waarmee het interval overlapt.\\
De methode \verb|BerekenSnijpunten(Cirkel c)|is dezelfde methode van in algoritme 1 en 2.
\subsubsection{Complexiteit}
Het complexiteit van dit algoritme is een benadering van de gevraagde $O((N+S)Log(N))$. Er gaan namelijk mogelijks een aantal onnodige controles op snijpunten worden uitgevoerd. De situatie waarin dit voorkomt, wordt in de volgende figuur gegeven. De vierkanten van de cirkels snijden elkaar wel maar de cirkels op zich niet.
\begin{figure}[H]
\centering
\includegraphics[width=0.55\textwidth]{onnodigecheck.png}
\end{figure}
Aangezien algoritme hard gelijkt op algoritme 2, kunnen we de complexiteit gelijkaardig berekenen. Opnieuw gaan we alle punten sorteren, hiervan weten we dat dit gebeurt in $O(N\cdot Log(N))$, net zoals in de vorige 2 algoritmen. Telkens waneer we nu een links punt tegenkomen van een lijnstuk van een cirkel, gaan we het interval opdit punt niet direct toevoegen aan de intervaltree (in tegenstelling tot algoritme 2 waarbij het punt wel direct aan een lijst wordt toegevoegd). Eerst controleren we met welke nodes in de intervaltree het interval overlapt. De duur van \'e\'en van deze operaties hangt af van de implementatie van de intervaltree. Als dit een gebalanceerde Red-Blacktree zou zijn dan is dit gegarandeerd in $O(Log(N))$; in onze orignele implementatie waren. Onze implementatie werkt echter als een ongebalanceerde boom. In worst case geeft dit $O(N)$. Het aantal keer dat we deze operatie moeten uit
\subsubsection{Randgevallen}
De sterkste beperking wordt opgelegd door de intervaltree. Er mogen namelijk geen twee exact dezelfde intervallen zijn die een zelfde laagste punt hebben (een interval bestaat uit een lo, laagste punt en een hi, het hoogste punt). Ook al zijn de intervallen misschien niet evenlang. 
\section{Correctheid van de algoritmen verifi\"eren}

Om de correctheid van de algoritmen te verifi\"eren hebben we verschillende stappen ondernomen. Deze stappen moeten in chronologische volgorde worden uitgevoerd omdat we assumpties uit voorgaande stappen gebruiken om te huidige stap te verifi\"eren.

\subsection{Enkele manuele testen}

\subsubsection*{bepalen dat cirkels snijden}
Onze eerste zorg was er zeker van zijn dat onze methode voor het bepalen dat cirkels snijden correct werkt. Met behulp van internet hebben we formules afgeleid om te bepalen of cirkels snijden. Na dat we deze formules afgeleid hebben, hebben we deze omgezet in Python code.

\subsubsection*{Bepalen van snijpunten van twee cirkels}
Na dat we er zeker van zijn dat onze cirkels snijden of niet moeten we bepalen waar deze snijden. Wederom hebben we met behulp van internet formules afgeleid om de snijpunten van circels te bepalen. Deze keer hebben we ook tussenstappen bepaald om te berekenen dat er \'e\'en of twee snijpunten zijn. Deze formules hebben we ook omgezet in Python code.

\subsubsection*{Algoritme 1}

Aangezien in algoritme \'e\'en alle cirkels met alle andere cirkels vergeleken worden gaan we zeker geen snijpunten missen. Verder hebben we ook al bewezen dat het bepalen dat cirkels snijden en de bepaling van eventuele snijpunten correct zijn. Uit deze twee stellingen kunnen we concluderen dat algoritme \'e\'en correct is.
Om eventuele bugs in de code uit te sluiten hebben we een klein aantal cirkels handmatig berekend. Dit resultaat hebben we vergeleken met het resultaat dat onze code ons geeft.

\subsubsection*{Algoritme 2}


Het is niet zo eenvoudig om te bewijzen dat algoritme twee correct is. De code voor het bepalen dat twee cirkels snijden en het berekenen van de snijpunten hergebruiken we. Op deze manier zijn we zeker dat hier geen fout inzit.
Aangezien algoritme \'e\'en correct is moeten we met algoritme twee dezelfde uitkomst krijgen als we dezelfde invoer gebruiken. Dit maakt het verifi\"eren van de correctheid al een stuk eenvoudiger.

Als eerst hebben we de uitkomst van algoritme \'e\'en en twee naast elkaar geplakt in een rekenblad. Nadat we beide uitkomstensets gesorteerd hebben, hebben we een paar steekproeven genomen om te controleren of deze dezelfde zijn. Dit bleek zo te zijn. In dit zelfde rekenblad hebben we ook de som genomen van de x-waarden en de y-waarden. Deze twee sommen blijken ook overeen te komen. Uit deze twee proeven kunnen we concluderen dat de resultaten voor beide algoritmen overeen komen.

Na deze numerieke test hebben we ook nog een visuele test uitgevoerd. De resultaten die de algoritmes terug geven zijn eenvoudig om te zetten naar een Scalable Vector Graphics bestand. In dit bestand hebben we eerst alle snijpunten die algoritme \'e\'en opleverd geschreven in \'e\'en bepaalde kleur. Daarna alle snijpunten die algoritme twee opleverd in een andere kleur. Aangezien enkel het laatst geschreven punt zichtbaar is, kunnen we makkelijk visueel controleren dat algoritme twee alle snijpunten vind door te controleren dat het geplotte svg bestand maar \'e\'en kleur weergeeft.

\begin{figure}[H]
\centering
\includegraphics[width=0.5\textwidth]{vb_svg.png}
\caption*{Voorbeeld van een svg die gegenereerd wordt}
\end{figure}

De numerieke en visuele test hebben we meermaals uitgevoerd met verschillende parameters voor de random gegenereerde cirkels om er zeker van te zijn dat onze testen niet toevallig slagen.

\subsubsection*{Algoritme 3}

De correctheid van algoritme drie gaan we op dezelde manier bepalen als algoritme twee. De code voor het bepalen dat cirkels snijden en het berekenen van snijpunten hergebruiken we wederom.

Als we de uitkomsten van algoritme \'e\'en en drie in een rekenblad plakken valt ons voor grote aantallen cirkels onmiddelijk op dat algoritme \'e\'en en drie niet hetzelfde aantal snijpunten vinden. Hieruit vermoeden we al dat algoritme drie niet correct werkt.

Ook de visuele test bevestigd ons vermoeden uit de numerieke test. De visuele test brengt wel een interessant gegeven aan het licht. Alle snijpunten die niet gevonden worden blijken in gebieden bij elkaar te liggen. (Zoals blijkt op onderstaande figuur.) Dit geeft ons het vermoeden dat er een fout zit in onze methode die de boom doorloopt opzoek naar snijdingen met het opgegeven segment.

\begin{figure}[H]
\centering
\includegraphics[width=0.5\textwidth]{gat_midden.png}
\caption*{Svg die een gebied bevat van snijpunten die niet worden gevonden}
\end{figure}

Bij nader onderzoek van de code komen we tot de vaststelling dat hoogstwaarschijnlijk onderstaand stuk code verantwoordelijk is voor de foutieve werking. Indien we \\ \verb|nodeQueue.put(current.left)| uitcommenten in de eerste if en commenten in de tweede if blijkt dat algoritme drie correct werkt. Op deze manier hebben we algoritme drie terug gereduceerd tot $O(n^{2})$ in het slechtste geval aangezien we elke cirkel terug gaan vergelijken met alle mogelijke cirkels die kunnen snijden.

\begin{verbatim}
if current.left is not None:
    #nodeQueue.put(current.left)
    if current.left.maxhi >= interval.lo.yco:
        nodeQueue.put(current.left)
                
\end{verbatim}
Uit dit onderzoek komen we tot de conclusie dat er takken van de zoekboom worden afgesneden die we wel moeten bezoeken. Dit wordt mogelijks veroorzaakt door een verkeerde maxhi waardoor er een cutoff ontstaat in de intervaltree waar er geen zou mogen zijn.

\section{Onderzoeksmethode}

We hebben voor elk algoritme de uitvoeringstijd en het aantal intersecties gemeten. De metingen lopen telkens tussen 20 en 1000 cirkels met stappen van 20. Om de resultaten zo correct mogelijk te houden hebben we voor verschillende parameters van de straal en het gebied waar de cirkels voorkomen gekozen.

Uit deze grafieken kunnen we de rekencomplexiteit van de algoritmen afleiden.

\subsection{Random cirkels tussen 0 en 1 met straal 0,1}
\begin{figure}[H]
\centering
\includegraphics[width=0.7\textwidth]{uitvoeringstijd_01.png}
\caption*{Uitvoeringstijd in functie van het aantal cirkels}
\end{figure}

\begin{figure}[H]
\centering
\includegraphics[width=0.7\textwidth]{snijpunten_01.png}
\caption*{Snijpunten in functie van het aantal cirkels}
\end{figure}

\subsection{Random cirkels tussen 0 en 1 met straal 0,5}
\begin{figure}[H]
\centering
\includegraphics[width=0.7\textwidth]{uitvoeringstijd_05.png}
\caption*{Uitvoeringstijd in functie van het aantal cirkels}
\end{figure}

\begin{figure}[H]
\centering
\includegraphics[width=0.7\textwidth]{snijpunten_05.png}
\caption*{Snijpunten in functie van het aantal cirkels}
\end{figure}

\subsection{Random cirkels tussen 0 en 1 met straal 1}
\begin{figure}[H]
\centering
\includegraphics[width=0.7\textwidth]{uitvoeringstijd_10.png}
\caption*{Uitvoeringstijd in functie van het aantal cirkels}
\end{figure}

\begin{figure}[H]
\centering
\includegraphics[width=0.7\textwidth]{snijpunten_10.png}
\caption*{Snijpunten in functie van het aantal cirkels}
\end{figure}

\subsection{Random cirkels tussen 0 en 1000 met straal 10}
\begin{figure}[H]
\centering
\includegraphics[width=0.7\textwidth]{uitvoeringstijd_100.png}
\caption*{Uitvoeringstijd in functie van het aantal cirkels}
\end{figure}

\begin{figure}[H]
\centering
\includegraphics[width=0.7\textwidth]{snijpunten_100.png}
\caption*{Snijpunten in functie van het aantal cirkels}
\end{figure}

\section{Bespreking experiment resultaten}

In de onderstaande bespreking, bespreken we algoritme drie zoals we het ge\"implementeerd hebben. Een belangrijke opmerking hierbij is dat algoritme drie niet correct werkt. In verschijdene pogingen hebben we opgemerkt dat het aantal niet herkende snijpunten van algoritme drie soms zeer hoog kan worden.

Wat ons onmiddelijk in alle gevallen opvalt is dat het aantal snijdingen kwadratisc toeneemt met het aantal cirkels.

\subsection{Random cirkels tussen 0 en 1 met straal 0,1}

Uit het eerste experiment blijkt duidelijk dat algoritme \'e\'en veel trager is dan algoritme twee. Algoritme drie heeft dezelfde trend als algoritme \'e\'en maar is toch net iets sneller. Uit de grafiek is ook duidelijk op te maken dat de algoritme \'e\'en en twee de theoretische rekencomplexiteit volgen. Algoritme drie wijkt af van de theoretische rekencomplexiteit door de foute implementatie.

De complexiteit van algoritme \'e\'en volgt ook perfect de kwadratische groei van het aantal snijpunten.

Algoritme twee ligt niet samen met algoritme \'e\'en omdat we niet in de worst-case van algoritme twee zitten. In dit specifieke geval zijn er steeds maar een klein aantal van het totaal aantal cirkels die met elkaar snijden. Hierdoor is algoritme twee toch nog relatief snel.

De pieken in de grafieken kunnen verklaard worden door het willekeurig generen van cirkels maar ook door andere processen die de CPU belasten alsook de garbage-collection van ons programma. 

\subsection{Random cirkels tussen 0 en 1 met straal 0,5}

Wederom blijkt dat algoritme \'e\'en het traagste is. Op de voet gevolgt door algoritme drie dat net iets beter presteerd. Algoritme twee is weer het snelste van de drie. Ook nu zien we dat algoritme \'e\'en en twee hun theoretische rekencomplexiteit volgen. Algoritme drie wijkt af van de theoretische rekencomplexiteit door de foute implementatie.

In dit geval ligt algoritme drie wel mooi tussen algoritme \'e\'en en twee. Bij grote aantallen cirkels valt ook het $O(N^{2})$ karakter van algoritme \'e\'en op ten opzichte van algoritme drie. De complexiteit van algoritme \'e\'en volgt werderom perfect de kwadratische groei van het aantal snijpunten.

De verwachte convergentie van algoritme twee naar algoritme \'e\'en blijft wel uit. Om deze te zien dienen we het aantal snijpunten nogmaals te verhogen.

De pieken in de grafieken kunnen verklaard worden door het willekeurig generen van cirkels maar ook door andere processen die de CPU belasten alsook de garbage-collection van ons programma. 

\subsection{Random cirkels tussen 0 en 1 met straal 1}

Bij deze grafiek springt direct de traagheid van algoritme drie in het oog. Algoritme drie is veel trager dan algoritme \'e\'en en twee. Deze traagheid komt waarschijnlijk door de foutieve implementatie van het algoritme.

Een ander opmerkelijk feit is de ligging van algoritme twee in de buurt van algoritme \'e\'en. Indien we even naar de parameters van de willekeurige cirkels kijken kunnen we concluderen dat er gigantisch veel snijpunten zijn in een klein oppervlak. Hierdoor gaat algoritme twee zich net als algoritme \'e\'en gedragen. De ietwat betere prestaties van algoritme twee zijn te verklaren door het feit dat algoritme twee de mogelijkheid heeft om een klein aantal cirkels over te slaan.

In deze grafiek vinden we weer de bevestiging dat algoritme \'e\'en zich volgens zijn theoretische tijdscomplexiteit gedraagt.

De pieken in de grafieken kunnen verklaard worden door het willekeurig generen van cirkels maar ook door andere processen die de CPU belasten alsook de garbage-collection van ons programma. 

\subsection{Random cirkels tussen 0 en 1000 met straal 10}

Wat ons bij deze grafiek onmiddelijk opvalt is het vlakke verloop van de grafiek van algoritme twee. Indien we naar de parameters van de cirkels kijken kunnen we concluderen dat er relatief weinig snijpunten gaan zijn. In dit specifieke geval gaan er steeds zeer weinig cirkels met elkaar vergeleken worden waardoor algoritme twee zeer snel is.

Algoritme \'e\'en houdt steevast zijn theoretische rekencomplexiteit van $O(N^{2})$ aan. Wat we ook verwachten.

Algoritme drie houdt zich mooi in het midden tussen algoritme \'e\'en en twee. Dit ligt in de lijn van de verwachten als we de eerste twee grafieken in acht nemen.

De pieken in de grafieken kunnen verklaard worden door het willekeurig generen van cirkels maar ook door andere processen die de CPU belasten alsook de garbage-collection van ons programma. 
\end{document}